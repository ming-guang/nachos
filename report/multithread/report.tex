%%%%%%%%%%%%%%%%%%%%%%%%%%%%%%%%%%%%%%%%%
% University Assignment Title Page 
% LaTeX Template
% Version 1.0 (27/12/12)
%
% This template has been downloaded from:
% http://www.LaTeXTemplates.com
%
% Original author:
% WikiBooks (http://en.wikibooks.org/wiki/LaTeX/Title_Creation)
%
% License:
% CC BY-NC-SA 3.0 (http://creativecommons.org/licenses/by-nc-sa/3.0/)
% 
% Instructions for using this template:
% This title page is capable of being compiled as is. This is not useful for 
% including it in another document. To do this, you have two options: 
%
% 1) Copy/paste everything between \begin{document} and \end{document} 
% starting at \begin{titlepage} and paste this into another LaTeX file where you 
% want your title page.
% OR
% 2) Remove everything outside the \begin{titlepage} and \end{titlepage} and 
% move this file to the same directory as the LaTeX file you wish to add it to. 
% Then add \input{./title_page_1.tex} to your LaTeX file where you want your
% title page.
%t
%%%%%%%%%%%%%%%%%%%%%%%%%%%%%%%%%%%%%%%%%
\title{Báo cáo đồ án}
%----------------------------------------------------------------------------------------
%	PACKAGES AND OTHER DOCUMENT CONFIGURATIONS
%----------------------------------------------------------------------------------------

\documentclass[12pt]{article}
\usepackage[T5]{fontenc}
\usepackage[utf8]{inputenc}
\usepackage[vietnamese,english]{babel}
\usepackage{amsmath}
\usepackage{graphicx}
\usepackage[colorinlistoftodos]{todonotes}
\usepackage{listings}
\usepackage{hyperref}
\usepackage{scrextend}
\hypersetup{
    colorlinks=true,
    linkcolor=blue,
    filecolor=magenta,      
    urlcolor=cyan,
}


\begin{document}

\begin{titlepage}

\newcommand{\HRule}{\rule{\linewidth}{0.5mm}} % Defines a new command for the horizontal lines, change thickness here

\center % Center everything on the page
 
%----------------------------------------------------------------------------------------
%	HEADING SECTIONS
%----------------------------------------------------------------------------------------

\textsc{\LARGE Đại học Khoa học tự nhiên}\\[1.0cm] % Name of your university/college
\textsc{\large Môn học: Hệ điều hành }\\[0.5cm] % Minor heading such as course title

%----------------------------------------------------------------------------------------
%	TITLE SECTION
%----------------------------------------------------------------------------------------

\HRule \\[0.4cm]
{ \huge \bfseries Đa chương}\\[0.2cm] % Title of your document
\HRule \\[1.5cm]
 
%----------------------------------------------------------------------------------------
%	AUTHOR SECTION
%----------------------------------------------------------------------------------------

\begin{minipage}{0.4\textwidth}
\begin{flushleft} \large
\emph{Thành viên nhóm 18:}\\
{\fontsize{10}{12}\selectfont Nguyễn Trần Minh Quang - 20127298}\\
{\fontsize{10}{12}\selectfont Lê Hoàng Long - 18127047}\\
{\fontsize{10}{12}\selectfont Lê Tiến Đạt - 21127569}\\
\end{flushleft}
\end{minipage}
~
\begin{minipage}{0.4\textwidth}
\begin{flushright} \large
\emph{Giảng viên:} \\
Phạm Tuấn Sơn % Supervisor's Name
\end{flushright}
\end{minipage}\\[1cm]

% If you don't want a supervisor, uncomment the two lines below and remove the section above
%\Large \emph{Author:}\\
%John \textsc{Smith}\\[3cm] % Your name

%----------------------------------------------------------------------------------------
%	DATE SECTION
%----------------------------------------------------------------------------------------

% I don't want day because it is English
% {\large \today}\\[2cm] % Date, change the \today to a set date if you want to be precise

%----------------------------------------------------------------------------------------
%	LOGO SECTION
%----------------------------------------------------------------------------------------

\includegraphics{image/logo/logo-hcmus.png}\\[1cm] % Include a department/university logo - this will require the graphicx package
 
%----------------------------------------------------------------------------------------

\vfill % Fill the rest of the page with whitespace

\end{titlepage}

\renewcommand*\contentsname{Mục lục}
\tableofcontents
\vfill


\section{Cài đặt hỗ trợ đa chương}
\subsection{Cài đặt phân trang cho lớp Addrspace}
Trong hệ điều hành Nachos lớp \textit{Addrspace} là lớp biểu diễn vùng bộ nhớ cho mỗi \textbf{Thread}.

Hiện theo cài đặt mặt định lớp này chỉ dùng các vùng đầu tiên trong \textit{bộ nhớ của máy ảo}, cũng như chưa hỗ trợ \textit{paging}.

\subsubsection{Quản lý phân trang (Paging)}

Để quản lý phân trang, nhóm em sử dụng lớp \textit{BitMap} được cài đặt sẵn trong Nachos, lớp này hỗ trợ quản lý đánh dấu và tìm những bit chưa đánh dấu rất phù hợp cho việc quản lý phân trang.

Qua đó nhóm em đã bắt đầu định nghĩa  \textbf{pages} là một biến toàn cục ở \textit{system.h}
\begin{lstlisting}[caption={Định nghĩa BitMap pages}, label={lst:vdcode}, language=C++]
extern BitMap *pages;
\end{lstlisting}

Với kích thước bằng với số trang mà Nachos mong muốn sử dụng.

\subsubsection{Sử dụng trang}
Trong lớp \textit{Addrspace}, sử dụng \textbf{TranslationEntry} nhằm mục đích lưu các thông tin để ánh xạ \textbf{bộ nhớ của process} với \textbf{bộ nhớ của máy ảo}.

Chỉ cần quan tâm field \textbf{physicalPage}, field này dùng để lưu id trang sử dụng, ta sẽ thay đổi từ sử dụng các trang đầu tiên:
\begin{lstlisting}[caption={method physicalPage ban đầu}, label={lst:vdcode}, language=C++]
for (i = 0; i < numPages; i++) {
    // ...
    pageTable[i].physicalPage = i;
    // ...
}
\end{lstlisting}

Sang việc tìm kiếm các trang đang trống và chọn nó:
\begin{lstlisting}[caption={method physicalPage sau chỉnh sửa}, label={lst:vdcode}, language=C++]
for (i = 0; i < numPages; i++) {
    // ...
    pageTable[i].physicalPage = pages -> Find();
    // ...
}
\end{lstlisting}

Sau đó dựa vào field trên và tính toán vị trí phù hợp ở \textbf{bộ nhớ của máy ảo}.

\subsection{Thêm id cho Thread}
Tạo một biến cục bộ đếm số thread đã tạo ở file \textbf{Thread.cc}, tăng và sử dụng nó để gán vào ID của \textit{Thread} mới được tạo.

\section{Cài đặt các syscall}
\subsection{Cài đặt syscall Exec}
Tạo một instance lớp \textit{Thread} gọi method \textbf{Thread\#Fork} chạy hàm \textbf{StartProcess} và trả về id của \textit{Thread} từ instance ban đầu được tạo.
\end{document}